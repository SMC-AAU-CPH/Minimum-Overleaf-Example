\documentclass{article}
\usepackage[utf8]{inputenc}

\title{Example}
\author{Cumhur Erkut}
\date{\today}

\usepackage{natbib}
\usepackage{graphicx}
\usepackage{listings}
\usepackage[colorinlistoftodos]{todonotes}



\begin{document}

\maketitle

\begin{abstract}
Your abstract.\todo{You can use \textit{italic in a comment}, like this.}\todo[inline]{Or add an inline comment-}
\end{abstract}

\section{Introduction}
There is a theory which states that if ever anyone discovers exactly what the Universe is for and why it is here, it will instantly disappear and be replaced by something even more bizarre and inexplicable.
There is another theory which states that this has already happened.

\begin{figure}[h!]
\centering
\includegraphics[scale=1.7]{universe}
\caption{The Universe}
\label{fig:universe}
\end{figure}

Its main formula is 

\begin{equation}
    E = mc^2 .
\end{equation}

How can we code this? 

\begin{lstlisting}[language=Python, caption=Python example]
import numpy as np
    
def incmatrix(genl1,genl2):
    m = len(genl1)
    n = len(genl2)
    M = None #to become the incidence matrix
    VT = np.zeros((n*m,1), int)  #dummy variable
    
    #compute the bitwise xor matrix
    M1 = bitxormatrix(genl1)
    M2 = np.triu(bitxormatrix(genl2),1) 

    for i in range(m-1):
        for j in range(i+1, m):
            [r,c] = np.where(M2 == M1[i,j])
            for k in range(len(r)):
                VT[(i)*n + r[k]] = 1;
                VT[(i)*n + c[k]] = 1;
                VT[(j)*n + r[k]] = 1;
                VT[(j)*n + c[k]] = 1;
                
                if M is None:
                    M = np.copy(VT)
                else:
                    M = np.concatenate((M, VT), 1)
                
                VT = np.zeros((n*m,1), int)
    
    return M
\end{lstlisting}

\section{Conclusion}
``I always thought something was fundamentally wrong with the universe'' \citep{adams1995hitchhiker}. But how about Engel's work \cite{Engel2020DDSP}?

\bibliographystyle{plainnat}
\bibliography{ddsp,references}
\end{document}
